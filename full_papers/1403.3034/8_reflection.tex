Since 2007, the Railway Verification Group at Swansea University
Computer Science has been working in collaboration with Invensys
Rail. The group is supported by eight academics. It adopts formal
techniques to railway systems. This setup has led to a rich process of
information exchange, by mutual visits and internships, as well as
regular meetings. Out of this collaboration, there have been several
successful research projects covering research into verification of
interlocking
programs~\cite{kanso08b,kansoThesis,james10b,james10a,lawrence11,james13b},
the verification of scheme plan
designs~\cite{MNRST12,MNRST12HVC,nfm2013,james13a}, and capacity
analysis~\cite{isobeMNR12}. These projects have involved many
specification formalisms including propositional logic, CSP, Timed
CSP, \CSPB, Scade, \CASL and Agda. Here, it appears that operational
based models such as the one we have provided in \CASL are easier for
railway engineers to follow and understand.

The methodology outlined in this paper has also been successfully
applied to the Invensys Rail Data Model~\cite{DataModel},
see~\cite{james14} for full details. In this context, the DSL
definition is provided by Invensys Rail in the form of UML class
diagrams with accompanying narrative. This DSL definition has been
completely designed by Invensys Rail without our support. Hence,
developing the informal DSL that is used as a starting point for our
methodology is clearly a task that can be undertaken purely within
industry. With regards to the formalisation of the DSL in \CASL, this
step was performed in a cyclic manner. Namely, the computer scientists
would suggest a formalisation at a meeting with Invensys Rail, and
then feedback would be provided on the formalisation. This process was
repeated until both groups were happy with the resulting
formalisation. Verification support (in terms of supporting lemmas)
was then developed purely by the computer scientists at Swansea. As
this step does not alter the DSL, the engineers were happy with the
approach.

Finally, the graphical front end to the OnTrack tool provides a view
to scheme plans that is of the same nature as representations used
within Invensys Rail. The railway engineers have seen the verification
process we propose and are confident that they could follow it thanks
to the automated nature of the process. They have expressed an
interest in the extension of OnTrack to allow for importing and
exporting of models using an industrial format such as the XLDL (XML
Layout Description Language) format used by Invensys Rail. This would
allow the toolset to be more easily integrated within current
development processes at Invensys Rail. We note that even though
Invensys Rail believe technologies such as the OnTrack toolset will
improve their development processes, due to concerns around the
integrity of automated verification tools, quality assurance of a
scheme plan design will, in the near future, still rely upon
traditional testing and inspection. Here, tool qualification is the
issue: our tools are not yet ``proven-in-use'', i.e., there is no
experience from previous industrial safety-critical projects
suggesting that our tools are ``correct''; alternatively, our tools
are not yet certified by a designated authority.

Finally, it remains future work to perform a systematic evaluation
into how usable the presented verification is for railway engineers.
To do this, one could carry out a pilot project. This would involve
thorough time measurement and documentation for all activities that
our approach incurs including installing, using, and integrating
OnTrack into the development process. Reflecting upon these results
would allow us to check if our approach is indeed feasible. On the
management level, such a pilot project would then have to be evaluated
as to whether it is an improvement with respect to current
practice. It should be both more effective, in that it allows one to
achieve better results than current practice, and more efficient,
i.e., it should offer a better cost/result ratio.

%%% Local Variables: 
%%% mode: latex
%%% TeX-master: "paper"
%%% End: 
