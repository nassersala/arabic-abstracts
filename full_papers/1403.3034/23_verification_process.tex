\subsection{Contribution: The Resulting Verification Process}

The result of applying our methodology is a toolset accommodating the
verification process illustrated in
Figure~\ref{fig:verification_process}. This process is undertaken
purely by the engineers in the domain. It follows three main
steps:

\begin{figure}[h]
  \centering
    \includegraphics[width=\linewidth]{images/verification_process.png}
   \caption{A verification process based on the designed tools.}
  \label{fig:verification_process}
\end{figure}

\paragraph{V1: Model Development Based on the Informal DSL} The first
(optional) step within industry is to outline or specify a design
informally. This step should be undertaken using the vocabulary
outlined within the informal DSL.\\

\paragraph{V2: Graphical Modelling} Next, the domain engineer can
encode their design using the graphical editor. Once encoded, the
engineer can automatically produce formal specifications ready for
verification. As the graphical editor contains constructs that are
from the informal DSL, training and learning costs are minimal.\\

\paragraph{V3: Verification} Finally, the formal specifications can be
verified using (for \CASL) the Heterogeneous Toolset,
\Hets~\cite{hets07}. Due to the domain specific lemmas developed
during the design process, verification is automated and ``Push
Button''.



%%% Local Variables: 
%%% mode: latex
%%% TeX-master: "paper"
%%% End: 
