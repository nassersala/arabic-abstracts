%% Here you can specify new commands and environments that you intend
%% to use. Using commands can make your document easier to write, read
%% and be more consistent.

%% An example could be
%% \newcommand{\CASL}{\textrm{\textsc{Casl}}\xspace}
%% This would define the command \CASL, that would produce the LaTeX
%% code '\textrm{\textsc{Casl}}' (with an appropriate space at the
%% end) each time it was used.


\usepackage{microtype}
\usepackage{fancybox}
\usepackage{float}
\usepackage{hetcasl}
\usepackage{amsmath,amssymb,amsthm,stmaryrd,mathrsfs}
\usepackage{subfloat}
\usepackage{rotating}
\usepackage{caption}
\usepackage{framed}
%%\usepackage{tikz-uml}
%%\usepackage{babel}
%%\usepackage{subfig}
\usepackage{subfigure}
%%\usepackage{ltablex}
\usepackage{url}
%%\usepackage{tikz}
\usetikzlibrary{arrows}
\usetikzlibrary{automata}
\usetikzlibrary{backgrounds}
\usetikzlibrary{calc}
\usetikzlibrary{fit}
\usetikzlibrary{shapes}
\usetikzlibrary{snakes}


\tikzset{sortTZ/.style ={}}
\tikzset{subsortTZ/.style = {draw, <-}}
\tikzset{isomorphicTZ/.style = {subsortTZ, <->}}
\tikzset{fixedTextHeightProblem/.style = {text height=1.5ex,text depth=.25ex}}

%%%%%%%%%%%%%%%%%%%%%%%%%%%%%%%%%%%%%%%%%%%%%%%%%%%%%%%%%%%%
%                                                          %
%                                                          %
% Railway macros                                           %
%                                                          %
%                                                          %
%%%%%%%%%%%%%%%%%%%%%%%%%%%%%%%%%%%%%%%%%%%%%%%%%%%%%%%%%%%%

%%%%%%%%%%%%%%%%%%%%%%%%%%%%%%%%%%%%%%%%%%%%%%%%%%%%%%%%%%%%
%% Customiseable Lengths
%%%%%%%%%%%%%%%%%%%%%%%%%%%%%%%%%%%%%%%%%%%%%%%%%%%%%%%%%%%%

\newcommand{\RWConnectorHalfHeight}{1mm}
\newcommand{\RWPointHeight}{10mm}
\newcommand{\RWPointHalfWidth}{10mm}
%% The lengths of the units attached to the point within the junction
\newcommand{\RWJunctionUnitLength}{10mm}
%% The gap vertical between the platform edge and the track
\newcommand{\RWPlatformGapFromTrack}{2mm}
%% The gap horizontal gap between the platform edge and the connector
\newcommand{\RWPlatformGapFromConnector}{4mm}
%% The vertical height of the platform
\newcommand{\RWPlatformHeight}{3mm}

\newcommand{\RWSemiLongPointHalfWidth}{20mm}
\newcommand{\RWLongPointHalfWidth}{30mm}

%%%%%%%%%%%%%%%%%%%%%%%%%%%%%%%%%%%%%%%%%%%%%%%%%%%%%%%%%%%%
%% Connectors
%%%%%%%%%%%%%%%%%%%%%%%%%%%%%%%%%%%%%%%%%%%%%%%%%%%%%%%%%%%%

% param 1 = name of connector (must not exist)
% param 2 = coordinate for center of connector
\newcommand{\RWConnector}[2]{
  \coordinate (#1) at #2;
  \draw ($(#1) +(0,-\RWConnectorHalfHeight)$) -- ($(#1) +(0,+\RWConnectorHalfHeight)$);
}

% param 1 = name of connector to draw at (must exist)
% param 2 = label to be used on connector
\newcommand{\RWLabelConnectorBelow}[2]{
  \node [anchor = north] at ($(#1) +(0,-\RWConnectorHalfHeight)$) {#2};
}

% param 1 = name of connector to label (must exist)
% param 2 = label to be used on connector
\newcommand{\RWLabelConnectorAbove}[2]{
  \node [anchor = south] at ($(#1) +(0,+\RWConnectorHalfHeight)$) {#2};
}

% param 1 = name of connector 1 (must exist)
% param 2 = name of connector 2 (must exist)
% param 3 = label to be used on linear unit
\newcommand{\RWLabelLinearUnitAbove}[3]{
  \node [anchor = south] at ($(#1)!.5!(#2) +(0,+\RWConnectorHalfHeight)$) {#3};
}

% param 1 = name of connector 1 (must exist)
% param 2 = name of connector 2 (must exist)
% param 3 = label to be used on linear unit
\newcommand{\RWLabelLinearUnitBelow}[3]{
  \node [anchor = north] at ($(#1)!.5!(#2) +(0,-\RWConnectorHalfHeight)$) {#3};
}

% param 1 = name of connector 1 (must exist)
% param 2 = name of connector 2 (must exist)
% param 3 = label to be used on linear unit
\newcommand{\RWLinearUnitAbove}[3] {
  \draw (#1) -- (#2);
  \RWLabelLinearUnitAbove{#1}{#2}{#3}
}
% param 1 = name of connector 1 (must exist)
% param 2 = name of connector 2 (must exist)
% param 3 = label to be used on linear unit
\newcommand{\RWLinearUnitBelow}[3] {
  \draw (#1) -- (#2);
  \RWLabelLinearUnitBelow{#1}{#2}{#3}
}
%%%%%%%%%%%%%%%%%%%%%%%%%%%%%%%%%%%%%%%%%%%%%%%%%%%%%%%%%%%%
%% Points
%%%%%%%%%%%%%%%%%%%%%%%%%%%%%%%%%%%%%%%%%%%%%%%%%%%%%%%%%%%%

%    /----
% --------
% param 1 = name for center of point (must not exist)
% param 2 = name for center of left connector (must not exist)
% param 3 = name for center of normal connector (must not exist)
% param 4 = name for center of reverse connector (must not exist)
% param 5 = coordinate for center of point
\newcommand{\RWPoint}[5]{
  \coordinate (#1) at #5;
  \RWConnector{#2}{($#5 + (-\RWPointHalfWidth,0)$)}
  \RWConnector{#3}{($#5 + (\RWPointHalfWidth,0)$)}
  \RWConnector{#4}{($#5 + (\RWPointHalfWidth,\RWPointHeight)$)}

  \draw (#2) -- #5;
  \draw #5 -- (#3);
  \draw #5 -- (#4);
}

\newcommand{\FigPoint}[5]{
  \coordinate (#1) at #5;
  \RWConnector{#2}{($#5 + (-30mm,0)$)}
  \RWConnector{#3}{($#5 + (\RWPointHalfWidth,0)$)}
  \RWConnector{#4}{($#5 + (\RWPointHalfWidth,\RWPointHeight)$)}

  \draw (#2) -- #5;
  \draw #5 -- (#3);
  \draw #5 -- (#4);
}

% --------
%    \----

% param 1 = name for center of point (must not exist)
% param 2 = name for center of left connector (must not exist)
% param 3 = name for center of normal connector (must not exist)
% param 4 = name for center of reverse connector (must not exist)
% param 5 = coordinate for center of point
\newcommand{\RWPointUpsideDown}[5]{
  \coordinate (#1) at #5;
  \RWConnector{#2}{($#5 + (-\RWPointHalfWidth,0)$)}
  \RWConnector{#3}{($#5 + (\RWPointHalfWidth,0)$)}
  \RWConnector{#4}{($#5 + (\RWPointHalfWidth,-\RWPointHeight)$)}

  \draw (#2) -- #5;
  \draw #5 -- (#3);
  \draw #5 -- (#4);
}

% ----\
% ---------
% param 1 = name for center of point (must not exist)
% param 2 = name for center of right connector (must not exist)
% param 3 = name for center of normal connector (must not exist)
% param 4 = name for center of reverse connector (must not exist)
% param 5 = coordinate for center of point
\newcommand{\RWPointReverse}[5]{
  \coordinate (#1) at #5;
  \RWConnector{#2}{($#5 + (+\RWPointHalfWidth,0)$)}
  \RWConnector{#3}{($#5 + (-\RWPointHalfWidth,0)$)}
  \RWConnector{#4}{($#5 + (-\RWPointHalfWidth,\RWPointHeight)$)}

  \draw (#2) -- #5;
  \draw #5 -- (#3);
  \draw #5 -- (#4);
}


% ---------
%     /----
% param 1 = name for center of point (must not exist)
% param 2 = name for center of right connector (must not exist)
% param 3 = name for center of normal connector (must not exist)
% param 4 = name for center of reverse connector (must not exist)
% param 5 = coordinate for center of point
\newcommand{\RWPointReverseUpsideDown}[5]{
  \coordinate (#1) at #5;
  \RWConnector{#2}{($#5 + (+\RWPointHalfWidth,0)$)}
  \RWConnector{#3}{($#5 + (-\RWPointHalfWidth,0)$)}
  \RWConnector{#4}{($#5 + (-\RWPointHalfWidth,-\RWPointHeight)$)}

  \draw (#2) -- #5;
  \draw #5 -- (#3);
  \draw #5 -- (#4);
}



%%%%%%%%%%%%%%%%%%%%%%%%%%%%%%%%%%%%%%%%%%%%%%%%%%%%%%%%%%%%
%% LongPoints
%%%%%%%%%%%%%%%%%%%%%%%%%%%%%%%%%%%%%%%%%%%%%%%%%%%%%%%%%%%%

%    /----
% --------
% param 1 = name for center of point (must not exist)
% param 2 = name for center of left connector (must not exist)
% param 3 = name for center of normal connector (must not exist)
% param 4 = name for center of reverse connector (must not exist)
% param 5 = coordinate for center of point
\newcommand{\RWLongPoint}[5]{
  \coordinate (#1) at #5;
  \RWConnector{#2}{($#5 + (-\RWPointHalfWidth,0)$)}
  \RWConnector{#3}{($#5 + (3*\RWPointHalfWidth,0)$)}
  \RWConnector{#4}{($#5 + (\RWPointHalfWidth,\RWPointHeight)$)}

  \draw (#2) -- #5;
  \draw #5 -- (#3);
  \draw #5 -- (#4);
}

% --------
%    \----

% param 1 = name for center of point (must not exist)
% param 2 = name for center of left connector (must not exist)
% param 3 = name for center of normal connector (must not exist)
% param 4 = name for center of reverse connector (must not exist)
% param 5 = coordinate for center of point
\newcommand{\RWLongPointUpsideDown}[5]{
  \coordinate (#1) at #5;
  \RWConnector{#2}{($#5 + (-\RWPointHalfWidth,0)$)}
  \RWConnector{#3}{($#5 + (3*\RWPointHalfWidth,0)$)}
  \RWConnector{#4}{($#5 + (\RWPointHalfWidth,-\RWPointHeight)$)}

  \draw (#2) -- #5;
  \draw #5 -- (#3);
  \draw #5 -- (#4);
}

% ----\
% ---------
% param 1 = name for center of point (must not exist)
% param 2 = name for center of right connector (must not exist)
% param 3 = name for center of normal connector (must not exist)
% param 4 = name for center of reverse connector (must not exist)
% param 5 = coordinate for center of point
\newcommand{\RWLongPointReverse}[5]{
  \coordinate (#1) at #5;
  \RWConnector{#2}{($#5 + (\RWPointHalfWidth,0)$)}
  \RWConnector{#3}{($#5 + (-\RWLongPointHalfWidth,0)$)}
  \RWConnector{#4}{($#5 + (-\RWPointHalfWidth,\RWPointHeight)$)}

  \draw (#2) -- #5;
  \draw #5 -- (#3);
  \draw #5 -- (#4);
}


% ----\
% ---------
% param 1 = name for center of point (must not exist)
% param 2 = name for center of right connector (must not exist)
% param 3 = name for center of normal connector (must not exist)
% param 4 = name for center of reverse connector (must not exist)
% param 5 = coordinate for center of point
\newcommand{\RWSemiLongPointReverse}[5]{
  \coordinate (#1) at #5;
  \RWConnector{#2}{($#5 + (\RWPointHalfWidth,0)$)}
  \RWConnector{#3}{($#5 + (-\RWSemiLongPointHalfWidth,0)$)}
  \RWConnector{#4}{($#5 + (-\RWPointHalfWidth,\RWPointHeight)$)}

  \draw (#2) -- #5;
  \draw #5 -- (#3);
  \draw #5 -- (#4);
}




% ---------
%     /----
% param 1 = name for center of point (must not exist)
% param 2 = name for center of right connector (must not exist)
% param 3 = name for center of normal connector (must not exist)
% param 4 = name for center of reverse connector (must not exist)
% param 5 = coordinate for center of point
\newcommand{\RWLongPointReverseUpsideDown}[5]{
  \coordinate (#1) at #5;
  \RWConnector{#2}{($#5 + (+\RWPointHalfWidth,0)$)}
  \RWConnector{#3}{($#5 + (-\RWLongPointHalfWidth,0)$)}
  \RWConnector{#4}{($#5 + (-\RWPointHalfWidth,-\RWPointHeight)$)}

  \draw (#2) -- #5;
  \draw #5 -- (#3);
  \draw #5 -- (#4);
}



%%%%%%%%%%%%%%%%%%%%%%%%%%%%%%%%%%%%%%%%%%%%%%%%%%%%%%%%%%%%
%% Junctions
%%%%%%%%%%%%%%%%%%%%%%%%%%%%%%%%%%%%%%%%%%%%%%%%%%%%%%%%%%%%

%    /----
% --------
% param 1 = name for center of point (must not exist)
% param 2 = name for center of left connector (must not exist)
% param 3 = name for center of normal connector (must not exist)
% param 4 = name for center of reverse connector (must not exist)
% param 5 = coordinate for center of point
% param 6 = label for linner unit 1 - left
% param 7 = label for linner unit 2 - normal
% param 8 = label for linner unit 3 - reverse
% param 9 = label for the point
\newcommand{\RWJunction}[9]{
  %% We assume RWTempA, RWTempB and RWTempC coordinate names are not
  %% used. These can be reused as they are just overwritten.
  \RWPoint{#1}{RWTempA}{RWTempB}{RWTempC}{#5} 
  \RWConnector{#2}{($(RWTempA) + (-\RWJunctionUnitLength,0)$)}
  \RWConnector{#3}{($(RWTempB) + (+\RWJunctionUnitLength,0)$)}
  \RWConnector{#4}{($(RWTempC) + (+\RWJunctionUnitLength,0)$)}
  \draw (RWTempA) -- (#2);
  \draw (RWTempB) -- (#3);
  \draw (RWTempC) -- (#4);
  \node[anchor = north] at ($(RWTempA)!.5!(#2)$) {#6};
  \node[anchor = north] at ($(RWTempB)!.5!(#3)$) {#7};
  \node[anchor = south] at ($(RWTempC)!.5!(#4)$) {#8};
  \node[anchor = north] at ($(RWTempA)!.5!(RWTempB)$) {#9};
}

% ----\
% ---------
% param 1 = name for center of point (must not exist)
% param 2 = name for center of right connector (must not exist)
% param 3 = name for center of normal connector (must not exist)
% param 4 = name for center of reverse connector (must not exist)
% param 5 = coordinate for center of point
% param 6 = label for linner unit 1 - right
% param 7 = label for linner unit 2 - normal
% param 8 = label for linner unit 3 - reverse
% param 9 = label for the point
\newcommand{\RWJunctionReverse}[9]{
  %% We assume RWTempA, RWTempB and RWTempC coordinate names are not
  %% used. These can be reused as they are just overwritten.
  \RWPointReverse{#1}{RWTempA}{RWTempB}{RWTempC}{#5}
  \RWConnector{#2}{($(RWTempA) + (+\RWJunctionUnitLength,0)$)}
  \RWConnector{#3}{($(RWTempB) + (-\RWJunctionUnitLength,0)$)}
  \RWConnector{#4}{($(RWTempC) + (-\RWJunctionUnitLength,0)$)}
  \draw (RWTempA) -- (#2);
  \draw (RWTempB) -- (#3);
  \draw (RWTempC) -- (#4);
  \node[anchor = north] at ($(RWTempA)!.5!(#2)$) {#6};
  \node[anchor = north] at ($(RWTempB)!.5!(#3)$) {#7};
  \node[anchor = south] at ($(RWTempC)!.5!(#4)$) {#8};
  \node[anchor = north] at ($(RWTempA)!.5!(RWTempB)$) {#9};
}

%%%%%%%%%%%%%%%%%%%%%%%%%%%%%%%%%%%%%%%%%%%%%%%%%%%%%%%%%%%%
%% Platforms
%%%%%%%%%%%%%%%%%%%%%%%%%%%%%%%%%%%%%%%%%%%%%%%%%%%%%%%%%%%%

% Make a platform between two connectors assuming the connectors have
% the same y values and that the first is positioned to the left of
% the second.
% A platform also joins the connectors with a track (a line)
% param 1 = name of the platform (must not exist)
% param 2 = name of the first connector (must exist)
% param 3 = name of the second connector (must exist)
\newcommand{\RWPlatformAbove}[3]{
  \draw[fill=gray] ($(#2) + (\RWPlatformGapFromConnector,\RWPlatformGapFromTrack)$) -- ($(#3) + (-\RWPlatformGapFromConnector,\RWPlatformGapFromTrack)$) -- 
    ($(#3) + (-\RWPlatformGapFromConnector,\RWPlatformGapFromTrack) + (0,\RWPlatformHeight)$) -- ($(#2) + (\RWPlatformGapFromConnector,\RWPlatformGapFromTrack) + (0,\RWPlatformHeight)$) --
    cycle;

  \coordinate (#1) at ($(#2)!.5!(#3) + (0,\RWPlatformGapFromTrack) + 0.5*(0, \RWPlatformHeight)$);
  \draw (#2) -- (#3);
}

% Make a platform between two connectors assuming the connectors have
% the same y values and that the first is positioned to the left of
% the second.
% A platform also joins the connectors with a track (a line)
% param 1 = name of the platform (must not exist)
% param 2 = name of the first connector (must exist)
% param 3 = name of the second connector (must exist)
\newcommand{\RWPlatformBelow}[3]{
  \draw[fill=gray] ($(#2) + (\RWPlatformGapFromConnector,-\RWPlatformGapFromTrack)$) -- ($(#3) + (-\RWPlatformGapFromConnector,-\RWPlatformGapFromTrack)$) -- 
    ($(#3) + (-\RWPlatformGapFromConnector,-\RWPlatformGapFromTrack) + (0,-\RWPlatformHeight)$) -- ($(#2) + (\RWPlatformGapFromConnector,-\RWPlatformGapFromTrack) + (0,-\RWPlatformHeight)$) --
    cycle;

  \coordinate (#1) at ($(#2)!.5!(#3) + (0,-\RWPlatformGapFromTrack) + 0.5*(0, -\RWPlatformHeight)$);
  \draw (#2) -- (#3);
}

% param 1 = name of the platform to label (must exist)
% param 2 = label to be used on platform
\newcommand{\RWLabelPlatformAbove}[2]{
  \node [anchor = south] at ($(#1) + 0.5*(0, \RWPlatformHeight)$) {#2};
}

% param 1 = name of the platform to label (must exist)
% param 2 = label to be used on platform
\newcommand{\RWLabelPlatformBelow}[2]{
  \node [anchor = north] at ($(#1) + 0.5*(0, -\RWPlatformHeight)$) {#2};
}


\renewcommand{\tt}{\texttt}
\renewcommand{\it}{\textit}
\renewcommand{\bf}{\textbf}

%%\newsubfloat{figure}

\usepackage{todonotes}
%%\newcommand{\todo}[1]{\bf{#1}}

\newcommand{\CASL}{\textrm{\textsc{Casl}}\xspace}

\newcommand{\ModalCASL}{\textrm{\textsc{ModalCasl}}\xspace }

\newcommand{\category}[1]{\ensuremath{\mathop{\text{{#1}}}}}
\newcommand{\functor}[1]{\ensuremath{\mathop{\text{{#1}}}}}
\newcommand{\functorsymbol}[1]{#1}
\newcommand{\ModReduct}{\ensuremath{\category{Mod\_Reduct}}\xspace}

\newcommand{\Sen}{\functor{sen}}
\newcommand{\ModFunctor}{\functor{mod}}

\newcommand{\PCFOL}{\text{\textit{PCFOL$^{=}$}}\xspace}
\newcommand{\SubPCFOL}{\text{\textit{SubPCFOL$^{=}$}}\xspace}
\newcommand{\ModSubPCFOL}{\text{\textit{ModalSubPCFOL$^{=}$}}\xspace}
\newcommand{\op}{op}
\newcommand{\inj}{\mbox{\texttt{inj}}\xspace}
\newcommand{\pr}{\mbox{\texttt{pr}}\xspace}

\newcommand{\angles}[1]{\langle #1 \rangle}

\newcommand{\partialto}{\rightarrow? \;}

\newenvironment{mybox}
{\begin{Sbox}\begin{minipage}{0.8\textwidth}}
{\end{minipage}\end{Sbox}\begin{center}\shadowbox{\TheSbox}\end{center}}

\newcommand{\Nat}{\mathbb{N}\xspace}

% these are (projection) functions
\newcommand{\Sig}{\ensuremath{\mathop{\text{\textbf{Sig}}}}}
\newcommand{\Mod}{\ensuremath{\mathop{\text{\textbf{Mod}}}}}




\newcommand{\DB}{Dines Bj{\o}rner\xspace}
\newcommand{\Bjoerner}{Bj{\o}rner\xspace}

\newcommand{\CSPB}{CSP$||$B\xspace}

\newcommand{\Hets}{\textrm{\textsc{Hets}}\xspace}
\newcommand{\SPASS}{SPASS\xspace}
\newcommand{\Vampire}{Vampire\xspace}
%% Tikz Styles
%\tikzstyle{category}     =[circle, draw, minimum size=3cm]
%\tikzstyle{categoryLarge}=[ellipse, draw, minimum width=10cm, minimum height=5.5cm]
%\tikzstyle{functor}      =[->, above, shorten <=0.2cm, shorten >=0.2cm]
%\tikzstyle{object}       =[]
%\tikzstyle{morphism}     =[->]



%% UML
\newcommand{\uml}[1]{\textsf{#1}}
\newcommand{\stereotype}[1]{\uml{\flqq#1\frqq}}
\newcommand{\aggregation}{\raisebox{0.2pt}{\begin{sideways}\fontsize{6pt}{6pt}\selectfont$\lozenge$\end{sideways}}}
\newcommand{\composition}{\raisebox{0.2pt}{\begin{sideways}\fontsize{6pt}{6pt}\selectfont$\blacklozenge$\end{sideways}}}


\newcommand{\Tau}{\mathrm{T}}
\newcommand{\Ypsilon}{\mathrm{Y}}
\newcommand{\omikron}{\ensuremath{o}}
%%\newcommand{\partialto}{\rightharpoonup}
\def\compfun{\mathbin{\circ}}
\def\reductop{\mathnormal{|}}
\def\sem#1{\mathopen\llbracket#1\mathclose\rrbracket}



%% Institutions
\newcommand{\institution}[1]{\mathscr{#1}}
\newcommand{\instSigop}{\mathrm{Sig}}
\newcommand{\instSig}[1]{\instSigop^{\institution{#1}}}
\newcommand{\instModop}{\mathrm{Mod}}
\newcommand{\instMod}[1]{\instModop^{\institution{#1}}}
\newcommand{\instSenop}{\mathit{Sen}}
\newcommand{\instSen}[1]{\instSenop^{\institution{#1}}}
\newcommand{\instmodels}[2][]{\mathrel{\models^{\institution{#2}}_{#1}}}
\newcommand{\instPresop}{\mathrm{Pres}}
\newcommand{\instPres}[1]{\instPresop^{\institution{#1}}}




\newcommand{\ippinstitution}[1]{\institution{\wp_{*}^{\mathrm{i}}#1}}
\newcommand{\ippinstSig}[1]{\instSig{\ippinstitution{#1}}}
\newcommand{\ippinstSen}[1]{\instSen{\ippinstitution{#1}}}
\newcommand{\ippinstMod}[1]{\instMod{\ippinstitution{#1}}}
\newcommand{\ippinstmodels}[2][]{\instmodels[#1]{\ippinstitution{#2}}}

\newcommand{\ippinstcomorph}[1]{\mu^{\ippinstitution{#1}, \institution{#1}}}
\newcommand{\ippinstcomorphSig}[1]{\Phi^{\ippinstitution{#1}, \institution{#1}}}
\newcommand{\ippinstcomorphSen}[1]{\alpha^{\ippinstitution{#1}, \institution{#1}}}
\newcommand{\ippinstcomorphMod}[1]{\beta^{\ippinstitution{#1}, \institution{#1}}}

\newcommand{\incppinstitution}[1]{\institution{\wp_{*}^{\hookrightarrow}#1}}
\newcommand{\incppinstSig}[1]{\instSig{\incppinstitution{#1}}}
\newcommand{\incppinstSen}[1]{\instSen{\incppinstitution{#1}}}
\newcommand{\incppinstMod}[1]{\instMod{\incppinstitution{#1}}}
\newcommand{\incppinstmodels}[2][]{\instmodels[#1]{\incppinstitution{#2}}}

\def\natto{\mathrel{\dot{\mathnormal{\to}}}}
\def\limp{\mathrel{\Rightarrow}}

 \newcommand{\ainj}[1]{\lceil #1 \rceil}
 \newcommand{\apr}[1]{\lfloor #1 \rfloor}
 \newcommand{\nil}{\langle\rangle}
\newcommand{\cinj}{{\mathit inj}}
\newcommand{\cpr}{{\mathit pr}}



\usepackage{listings}

\def\Bpre{\;\mbox{\bf PRE }}
\def\Blet{\;\mbox{\bf LET}\;}
\def\Bbe{\;\mbox{\bf BE}\;}
\def\Bin{\;\mbox{\bf IN}\;}
\def\Bif{\;\mbox{\bf IF}\;}
\def\Bthen{\;\mbox{\bf THEN}\;}
\def\Bend{\;\mbox{\bf END}}
\def\Belse{\;\mbox{\bf ELSE}}
\def\BConstants{\;\mbox{\bf CONSTANTS}}
\def\BProperties{\;\mbox{\bf PROPERTIES}}
