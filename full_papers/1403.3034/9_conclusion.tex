In this paper, we have introduced a novel design methodology for
encapsulating formal methods within DSLs. We have supported our
hypothesis that DSLs can aid with verification by showing it to be
valid for the railway domain. 

Our methodology begins with industrial documents describing domain
elements in the form of UML class diagrams with narratives. This
informal DSL is formalised in the algebraic specification
language~\CASL, where we support this step with automated translations
of the UML class diagram, thus ensuring ``faithful modelling'' of the
domain. Next, systematic exploitation of domain knowledge allows us to
prove domain specific lemmas. Thanks to these domain specific lemmas,
properties on the formal specifications can automatically be proven
for a class of systems, thus addressing issues surrounding
``scalability''. Finally, a graphical editor for the DSL is
developed. This editor allows engineers from the domain to follow a
seamless verification process: (1) they can formulate models in the
DSL; (2) these models can be automatically transformed into formal
specifications; and (3) they can automatically verify these models
thanks to the domain specific lemmas. Overall, this addresses the
issues of ``accessibility''of formal methods.

The verification of railway control software has been identified as a
grand challenge of computer science~\cite{topic11}. The defining
element of a grand challenge is that progress towards the challenge
results in progress in computer science in general. Therefore,
motivated by the railway domain, our methodology can be seen as a step
forward for industrial applications of formal methods.

Along with presenting this methodology, we have also successfully
illustrated the use of algebraic specification for modelling railway
systems and the use of automated theorem proving for railway
verification. Bringing these points together results in a strong case
for using DSLs in the setting of specification and verification.

In the future, we would like to explore visual feedback of failed
proof attempts. Currently, feedback to the user is provided in the
form of a named route which is unsafe (obtained from the name of the
failed proof). Such visualisations have been considered by Marchi et
al.\ \cite{ic2011}, and it would be interesting to extend OnTrack with
such visualisations. We would also like to illustrate the
applicability of our methodology to further domains, such as to the
design of medical devices as considered by Oladimeji et
al.\ \cite{olad13}.
\\
\paragraph{Acknowledgements:}
The authors would like to thank Simon Chadwick and Dominic Taylor from
Invensys Rail UK for their contributions and encouraging feedback. We
would like to thank our colleagues from the Swansea railway
verification group and the Swansea Processes and Data research group
for their input and feedback towards this work. Similarly, we greatly
appreciate the input of Alexander Knapp and Till Mossakowski towards
our work on the UML institution and comorphism to \ModalCASL. We also
thank Helen Treharne, Steve Schneider and Matthew Trumble from Surrey
University for their collaboration in developing the OnTrack tool. Our
final thanks goes to Erwin R.\ Catesbeiana (Jr) for signalling us in
the correct direction.

%%% Local Variables: 
%%% mode: latex
%%% TeX-master: "paper"
%%% End: 
