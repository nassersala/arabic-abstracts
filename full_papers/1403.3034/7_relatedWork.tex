At this point, we reflect on other DSL based approaches for railway
verification, commenting on how they differ to our presented
methodology.

The Railway Control Systems Domain language (RCSD) is a DSL for
railway control systems developed by Kirsten
Mewes~\cite{mewes2010}. RCSD is motivated by model-driven engineering
approaches to system design. The language uses common domain notation
for its concrete syntax and incorporates knowledge from domain
engineers about the domain into its static and dynamic
semantics. Mewes also considers the topic of testing models created in
DSL. Here, domain specific constraints are included into the models
ensuring that validation checks for correct functionality can be
tested at the model level, before any software is been
developed. Mewes' work focuses on the design of RCSD. In contrast, we
leave the design of the DSL to the domain engineer and focus on
capturing domain knowledge that can be exploited for automatic
verification.

Work by Haxthausen and Peleska~\cite{peleska07} has also explored the
development of a domain specific framework for automated construction
and verification of railway control systems. Their framework consists
of a three tiered approach: the top layer is a DSL for use by domain
engineers to specify railway control systems; the second tier is model
generation: A generator automatically produces the model of a control
program based on the specification given by the design
engineer. Bounded model checking can then be performed to establish
various safety properties over such programs; the third tier allows
actual code to be automatically generated and verified to ensure
certain properties are maintained throughout process. Differing to our
work, this framework is specifically developed for the railway domain
and tied to a specific DSL. In this paper, we provide a generic and
systematic methodology, where the railway domain serves for
illustration.

Finally, the SafeCap toolset~\cite{iliasov2012b} provides a tooling
platform that supports reasoning about railway capacity whilst
ensuring system safety. It is based upon the SafeCap DSL, which
captures track topology, route and path definitions and signalling
rules. Overall, the toolset allows signalling engineers to design
stations and junctions, to check their safety and to evaluate the
potential improvements in capacity. Again, the SafeCap approach is
bound to a specific DSL, where the design of the DSL is tailored to
the underlying verification technology (Event B). This differs from
our aim to decouple the DSL from the formal specification language, in
turn allowing tooling environments to be very openly extendable.

%%% Local Variables: 
%%% mode: latex
%%% TeX-master: "paper"
%%% End: 
