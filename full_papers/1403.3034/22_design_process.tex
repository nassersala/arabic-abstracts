\subsection{Contribution: The Design Steps of our Methodology}

To achieve a faithful, scalable and accessible modelling and
verification procedure, we present a new methodology that encapsulates
formal methods within a DSL. This results in a tool based framework
for verification.

Considering Figure~\ref{fig:process}, the encapsulation process we
propose is undertaken by a team comprising of computer scientists
working in close collaboration with experts from the domain. Here, the
close working relationship ensures the resulting domain modelling is
faithful. The following steps are involved in the process:

\begin{figure}[h]
  \centering
  \includegraphics[width=\linewidth]{images/methodology.png}
  \caption{The proposed methodology.}
  \label{fig:process}
\end{figure}

\paragraph{M1: Formalising (Industrial) DSLs}
The starting point for our methodology is an informal domain
description in the form of UML Class Diagrams and accompanying
narrative. From such UML class diagrams, names, relations, and
multiplicity constraints can be automatically extracted and translated
into a formal specification in \ModalCASL. We suggest and support a
particular automatic translation (see
Section~\ref{sec:comorph}). Next, domain experts and computer
scientists extend the resulting formal \ModalCASL specification with a
modelling of the narrative that captures dynamics. This is possible as
\ModalCASL allows the description of state changes in terms of modal
operators. Here, we also encode proof goals for verification. Then,
for the sake of better proof support, we apply an existing automatic
translation from \ModalCASL to \CASL. Overall, starting from an
existing DSL and involving domain experts ensures a faithful
formalisation of the selected domain concepts.\\

\paragraph{M2: DSL Analysis for Verification Support}
The result of Step M1, the formalisation of the DSL, is a loose \CASL
specification -- see Section \ref{ssec:spec_languages} for
details. The logical closure of this specification, i.e., all theorems
that one can prove from this \CASL specification, is what we call the
implicitly encoded ``domain knowledge'' of the DSL. We make (part of)
this domain knowledge explicit in the form of lemmas that allow one to
refactor any proof goals into equivalent ones that are expressed on
the right level of granularity. Naturally, there cannot be a universal
solution to finding such domain specific lemmas. However, in our
experience, for all DSLs we have considered, such lemmas have existed,
follow from knowledge of the domain experts, and allow refactoring. We
discuss such lemmas in Section~\ref{sec:dsl_lemmas} and show that
these lemmas allow for scalable verification based on ideas that are
often inherent to the domain. Overall, this step enables scalability
of the verification approach.\\

\paragraph{M3: Graphical Tool Support}

DSLs are often accompanied by a development framework. For this we
make use of the \emph{Graphical Modelling Framework},
GMF~\cite{gronback09}. GMF provides the infrastructure to create, from
a UML class diagram, a Java based graphical editor. Using GMF, domain
experts and computer scientists create such a graphical tooling
environment for the DSL. This allows for native graphical
representations of domain elements. Such an editor is open (via
Epsilon~\cite{kolovos2012}) to extension with model
transformations~\cite{kolovos2012}. Such transformations allow for the
graphical models produced by the editor to be translated to \CASL
specifications. These specifications can be enriched with the domain
knowledge developed in Step M2. We illustrate this approach in
Section~\ref{sec:ontrack}, giving details of the OnTrack Toolset for
the railway domain. The result of this step is a tool for generation
of formal models that is readily usable by engineers from the domain
under consideration.

\subsubsection{Addressing the Issues}

Overall, our methodology addresses the issues we started with:
faithful modelling is achieved thanks to starting with an informal
description and forming a formal specification in a close working
relationship between the domain experts and computer scientists;
scalability of the verification procedure is achieved thanks to the
property supporting domain specific lemmas; accessibility to modelling
and verification of systems is achieved thanks to graphical tooling
incorporating domain specific concepts and constructs.

%%% Local Variables: 
%%% mode: latex
%%% TeX-master: "paper"
%%% End: 
