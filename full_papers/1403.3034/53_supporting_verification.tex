\subsection{Contribution: Developing DSL Specific Knowledge}
\label{sec:dsl_lemmas}

Within the railway domain, it is understood that control tables are
vital in ensuring safety. In our presentation we can see that movement
authorities are extended depending on the rules of the control
table. That is, for a movement authority to be extended by the regions
of a route, that route must be open. Through domain analysis, we can
reduce the reasoning on the level of movement authorities, to a
reasoning on the level of topological routes and the control
table. This is captured by following domain specific lemma:

%% That is, under the condition that the predicate
%% \IdDeclLabel{\Id{ext}}{ext} is encoded faithfully for the given model,
%% i.e. that for all movement authorities $m1, m2$ and routes $r$
%% $ext(m1,r, m2)$ holds if $m2$ is the extension of $m1$ with the
%% regions of $r$, given a scheme plan exhibits the following property:
%% \begin{center}
%% \textit{for all routes, if a region of that route is assigned at a
%%   particular time, then the route is not open at that time,}
%% \end{center}
%% then:
%% \begin{center}
%%  \textit{for all time, two overlapping movement
%%   authorities are assigned at the same time.}
%% \end{center}

%% \noindent This is captured formally by the following lemma:
\pagebreak
\begin{lem}[Property Reduction to Routes]
\label{lem:dsl}
  Given a scheme plan $SP$ then
\begin{small}
\begin{hetcasl}
\> \Ax{\forall} \Id{t} \Ax{:} \Id{Time}; \Id{m1}, \Id{m2} \Ax{:} \Id{MA}
\Ax{\bullet} \IdApplLabel{\Id{share}}{share}(\Id{m1}, \Id{m2})\\
\>\> \Ax{\Rightarrow} \Id{m1} \Ax{=} \Id{m2} \Ax{\vee} \Ax{\neg} (\IdApplLabel{\Id{assigned}}{assigned}(\Id{m1}, \IdApplLabel{\Ax{t}}{t}) \Ax{\wedge} \IdApplLabel{\Id{assigned}}{assigned}(\Id{m2}, \IdApplLabel{\Ax{t}}{t})) \`{\small{}\KW{\%}(*)\KW{\%}} 
\end{hetcasl}
\end{small}
\noindent if and only if,
\begin{small}
\begin{hetcasl}
\> \Ax{\forall} \Id{t} \Ax{:} \Id{Time}; \Id{r} \Ax{:} \Id{Route}; \Id{rg} \Ax{:} \IdApplLabel{\Id{Region}}{Region}; \Id{ma} \Ax{:} \Id{MA} 
\\ 
\>\>\Ax{\bullet} \IdApplLabel{\Id{assigned}}{assigned}(\Id{ma}, \Id{t}) \Ax{\wedge} \Id{rg} \IdApplLabel{\Id{eps}}{::eps::} \Id{ma} \Ax{\wedge} \Id{rg} \IdApplLabel{\Id{eps}}{::eps::} \IdApplLabel{\Id{regions}}{regions}(\Id{r})\\
\>\>\> \Ax{\Rightarrow} \Ax{\neg} \Id{r} \IdApplLabel{\Id{isOpenAt}}{::isOpenAt::} \Id{t}  \`{\small{}\KW{\%}(**)\KW{\%}} 
\end{hetcasl}
\end{small}
\end{lem}


%%\end{lemma}

The proof follows by induction on time from the axioms we have
presented, including the induction axiom of Section
\ref{ssec:failed_verification}. The full proof is given in
Appendix~\ref{app:proof}. We first completed this proof by hand, and
then attempted it with \Hets. Encoding the proof for automated theorem
proving led us to consider the role of empty routes in more depth than
in the hand written proof where we had assumed certain
details. See~\cite{james14} for details on the encoding of the proof
into steps within \CASL and how these steps allow the proof to be
automatically discharged using \Hets.

The result of this lemma is that we can reduce the verification
problem over movement authorities to a simpler problem over route
openness. At this point, we note that this lemma is completely
independent of any concrete scheme plan formulated using our extended
DSL. Hence, once it has been proven as a consequence of the DSL, it
can be used to aid with verification of any scheme plan formulated
using the extended DSL. In Section~\ref{ssec:verification} we show
that this lemma is enough to make verification of large scheme plans
highly feasible. 

Finally, Appendix~\ref{app:struture} highlights, mainly for clarity,
the specification structure that we have employed throughout our
modelling.


%%% Local Variables: 
%%% mode: latex
%%% TeX-master: "paper"
%%% End: 
