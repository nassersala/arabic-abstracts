\subsection{Background: Industrial Practice in the Railway Domain}\label{ssec:industry_rw}

In industry, companies such as our industrial partner, Invensys Rail,
undertake domain engineering with the aim to ``describe all concepts,
components and properties within the railway
domain''~\cite{DataModel}. This modelling, for example, includes
features such as rail topology (the basic graph underlying the
railway), dimensions (e.g, where tracks are with regards to reference
points, the length of tracks, etc), and signalling (routes, speed
restrictions etc.). Common across all these layers is the notion of a
\emph{track plan} as a term to describe layouts of junctions and
stations. Track plans combine topological information and the
conceptual abstraction of routes, which determine the use of a rail
layout.
% Track plans are part of design processes prescribed by Railway
% Authorities, such as Network Rail's \emph{Governance for Railway
%   Investment Projects} (GRIP) process. The first four phases of the
% GRIP process define the track plan and routes of the railway to be
% constructed. While phase five -- the detailed design -- is contracted
% to a signalling company such as Invensys. It is then the signalling
% company which chooses appropriate track equipment, adds control
% information to the track plan, and implements concrete control systems
% for running the railway.
An example track plan is shown in Figure~\ref{fig:trackplan}. 

\begin{figure}[h]
%\begin{footnotesize}
  \centering
\begin{tikzpicture}[transform shape]
 \node at (-0.5,0) {X};
 \node at (8.5,0) {Y};
 \draw [->] (0,0.7) to (1,0.7);
 \RWConnector{a}{(0,0)}
 \RWLabelConnectorBelow{a}{c1}
 \RWConnector{a1}{(1,0)}
 \RWLabelConnectorBelow{a1}{c2}
 \RWLinearUnitAbove{a}{a1}{LA1}
 \RWPoint{p1}{p1cl}{p1cn}{p1cr}{(2,0)}
 \RWLabelLinearUnitBelow{p1cn}{p1cl}{P1}
 \RWLabelConnectorBelow{p1cn}{c5}
 \RWConnector{a3}{(5,0)}
 \RWLabelConnectorBelow{a3}{c6}
 \RWLinearUnitAbove{p1cn}{a3}{PLAT2}
 \RWPointReverse{p2}{p2cl}{p2cn}{p2cr}{(6,0)}
 \RWLabelLinearUnitBelow{p2cn}{p2cl}{P2}
 \RWConnector{b1}{(3,1)}
 \RWLabelConnectorBelow{b1}{c3}
 \RWConnector{b2}{(5,1)}
 \RWLabelConnectorBelow{b2}{c4}
 \RWLinearUnitAbove{b1}{b2}{PLAT1}
 \RWConnector{b4}{(7,0)}
 \RWLabelConnectorBelow{b4}{c7}
 \RWConnector{b5}{(8,0)}
  \RWLabelConnectorBelow{b5}{c8}
 \RWLinearUnitAbove{b4}{b5}{LA2}
 \end{tikzpicture}              
\\ \quad \\
\begin{tabular}{|l|c|c|c|}
\hline
Route & Clear        & Normal & Reverse \\ \hline \hline
RX1   & LA1, P1, PLAT1 &      & P1        \\
R1Y   & P2, LA2      &     & P2        \\
RX2   & LA1, P1, PLAT2 & P1       &       \\
R2Y   & P2, LA2      & P2       &       \\ \hline
\end{tabular}
\begin{tabular}{|l|c|}
\hline
Route & Point(Cleared By)               \\ \hline \hline
RX1   & P1(P1)                         \\ 
R1Y   & P2(LA2)                         \\ 
RX2   & P1(P1)                         \\ 
R2Y   & P2(LA2)                         \\ \hline 
\end{tabular}\\ \quad \\
%\end{footnotesize}
\caption{A scheme plan for a simple station. Top: Track
  plan, Bottom Left: Control table, Bottom Right: Release table.}
\label{fig:trackplan}
\end{figure}


The intended operation of the train station shown in
Figure~\ref{fig:trackplan} is: (1) trains enter at X using track LA1,
they then proceed across point P1 towards the upper line to platform
PLAT1 (i.e.\ taking route RX1); (2) alternatively, they pass over
point P1 towards the lower line and proceed to platform PLAT2 (i.e.\
taking route RX2); (3) trains from platform PLAT1 can then pass back
the lower line using point P2 and exit the station through track LA2
(i.e.\ taking route R1Y); (4) finally, trains from platform PLAT2 can
pass across point P2 and exit the station through track LA2 (i.e.\
taking route R2Y).

Such a track plan is usually paired with a set of control and release
tables~\cite{kerr01} to form a scheme plan -- see Figure
\ref{fig:trackplan}. A scheme plan details the conditions required for
route availability. The operational setting and unsetting of points
and routes is controlled by an interlocking which is implemented based
on this scheme plan.

The control table prescribes that a given route can be used when all
tracks in the ``clear'' column are not occupied by a train, and the
points in the ``normal'' and ``reverse'' columns are set to those
positions. The example control table prescribes that route RX1 can be
assigned to a train when units LA1, P1, and PLAT1 are unoccupied and
point P1 is in its reverse position, that is, allowing trains to
travel to the top line of Figure~\ref{fig:trackplan}. We note that the
track plan in Figure~\ref{fig:trackplan} is uni-directional and that
if it were bi-directional, there would be routes corresponding to the
opposite direction of travel. The rules for these routes would also
share tracks with the current rules, stopping the possibility of
routes in opposite directions being used at the same time.

The interlocking also allocates so called locks on points to
particular route requests. These locks ensure that the point remains
locked in position. Such locks are then released according to the
information in the release table. For example, the first row of the
release table states that for route RX1 the point P1 can be released
by the point itself becoming clear. Releasing of these locks allows
the corresponding points to be used within another route. Notice, that
for a point which splits two routes, i.e.\ point P1, the lock on the
point can be released by the point itself. However, for points that
merge two routes, i.e.\ point P2, there is the added safety check that
the point can only be released after the shared parts of the routes
are cleared, i.e.\ track LA2. 

%% In this paper, we refrain from considering the concrete implementation
%% of an interlocking, although such a task has been considered by
%% several
%% others~\cite{james10a,kanso08b,fokkink98,peleska04,peleska07}. Instead,
%% we focus on verification of the control logic captured by the scheme
%% plan. The checking of the logic of these tables with respect the
%% topology of the track plan is vital in avoiding train
%% collisions. Later in Section~\ref{XXX} we will see that this forms the
%% basis of our verification problem.

Finally, the last element in the dynamic operation of railways is that
of train movements. Above, we have described how access to certain
routes is granted, and areas of tracks can be released. This follows
conventional railway signalling~\cite{kerr01}. However, the newer
ETCS~\cite{etcs} standard builds on this conventional signalling with
the notion of a movement authority. A movement authority can be
thought of as an area of railway for which a train can be granted
access to travel along. For example, a train may be granted access to
move along units LA1 and P1 in Figure~\ref{fig:trackplan}. The
assignment of movement authorities is given by the following
narratives:
\begin{description}
\item[N1 -- Extension] Initially, no train has a movement
  authority. A request can be made for a train to travel along a
  particular route. If the route is available (as dictated by the
  control table) then the train's movement authority can be extended to
  include the route.  The movement authority for a train can contain
  multiple routes. 
\item[N2 -- Release] As a train travels it releases regions of railway
  from its movement authority according to the release table for that
  route.
\end{description}
Such movement authorities allow the example run illustrated in
Figure~\ref{fig:example_movements}. In the beginning, at time $0$,
there is no train in the system. At time $1$, train A has been
detected by track circuit $LA1$. Train A travels to platform PLAT1
where it resides until it has been overtaken by train B. Train A then
travels further and leaves the system. This run illustrates that train
A releases the use of point P1 before it exits the full route
RX1. This allows for train B to use route RX2 whilst the end of route
RX1 is still in use by train A.


\begin{figure}[h]
  \centering
\begin{footnotesize}
\begin{tabular}{l|c|c|c|c|c|c}
Time & LA1 & P1 & PLAT1 & PLAT2 & P2 & LA2 \\ \hline \hline
0 & \_ & \_ & \_ & \_ & \_ & \_ \ \\
1 & $\Longmapsto_a$ & \_ & \_ & \_ & \_ & \_ \\
2 & \_ & $\Longmapsto_a$ & \_ & \_ & \_ & \_ \\
3 & \ & \_ & $\Longmapsto_a$ & \_ & \_ & \_ \\
4 & $\Longmapsto_b$ & \_ & $\Longmapsto_a$ & \_ & \_ & \_ \\
5 & $\_$ & $\Longmapsto_b$ & $\Longmapsto_a$ & \_ & \_ & \_ \\
6 & $\_$ & $\_$ & $\Longmapsto_a$ & $\Longmapsto_b$ & \_ & \_ \\
7 & $\_$ & $\_$ & $\Longmapsto_a$ & \_ & $\Longmapsto_b$ & \_ \\
8 & $\_$ & $\_$ & $\Longmapsto_a$ & \_ & \_ & $\Longmapsto_b$ \\
9 & $\_$ & $\_$ & $\Longmapsto_a$ & \_ & \_ & \_ \\
10 & $\_$ & $\_$ & \_ & \_ &  $\Longmapsto_a$ & \_ \\
11 & $\_$ & $\_$ & \_ & \_ & \_ & $\Longmapsto_a$ \\
12 & \_ & \_ & \_ & \_ & \_ & \_ \ \\
\end{tabular}
\end{footnotesize}
\caption{A time/position diagram for an example run of the station.}
  \label{fig:example_movements}
\end{figure}

\subsubsection{Discussion of Safety Properties}
\label{sec:safety_discussion}

In railway signalling, many different safety properties have been
considered. For example, on the concrete level of an interlocking, one
may want to check the concrete property that ``Signal x only shows
green when route y is free to use''~\cite{james10b}. Alternatively,
when checking the design of a scheme plan, Moller et al.\ check that
the control table ensures collision-freedom (excluding two trains
occupying the same track)~\cite{MNRST12HVC}. Our aim differs slightly,
as we consider the assignment of movement authorities which is
outlined by the ETCS standard~\cite{etcs}. Therefore we verify that
 ``overlapping movement authorities are not assigned at the same
  time''.

Such a property is at a higher level of abstraction than the
properties mentioned previously. However, as movement authorities are
extended depending on the rules of the control table, we do in fact
cover the property of collision freedom under the assumption that
trains are well behaved. That is, if trains stay within their given
movement authority, and movement authorities are proven to never
overlap, then we know two trains cannot occupy the same track unit.


%%% Local Variables: 
%%% mode: latex
%%% TeX-master: "paper"
%%% End: 
