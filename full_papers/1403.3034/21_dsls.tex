\subsection{Background: Domain Specific Languages (DSLs)}

Throughout all areas of science, one can find approaches that are
general in principle or specific to the task at hand. A general
approach gives a solution to several problems of a similar
manner. Whereas a specific approach often solves problems in a more
comprehensive manner, but can be applied to significantly less
problems. In computer science, the differences are exemplified by
general purpose languages (GPLs) and domain specific languages (DSLs)
respectively.

Domain specific languages (DSLs)~\cite{heering05}, are languages that
have been designed and tailored for a specific application
domain. DSLs aim to abstract away technical details of computer
science from the user, allowing them to create programs or
specifications without having to be an expert programmer or
specifier. Examples of DSLs include the well known Backus Naur Form or
the commonly used HTML markup language. Considering HTML, it is
designed explicitly with webpage creation in mind. It has specific
features such as \textit{elements}, \textit{tags} and
\textit{attributes} that allow the specification of structure within a
web page. The advantages of having these domain specific features are
apparent as HTML has become the de facto standard for webpage creation
thanks to its expressiveness and to its ease of application. When we
speak of expressiveness in the context of DSLs, we refer to the
question of how easy it is for a user to describe the objects they
desire. Expressiveness can often be explored by considering how
intuitive various language constructs are. This idea forms a main
theme throughout this work.

\subsubsection{DSLs Formulated using UML Class Diagrams and Narrative}

UML Class Diagrams~\cite{uml} are industrially accepted for modelling
a variety of systems across numerous domains. Often they are used to
describe all elements and relationships occurring within a domain. As
such, a UML Class Diagram can be thought of as describing a DSL. Many
tools and frameworks actually use UML class diagrams as a starting
point for the description of a DSL~\cite{gronback09,kolovos2012}. A
typical example of such an endeavour is given by the Data
Model~\cite{DataModel} of our research partner Invensys
Rail\footnote{www.invensys.com}. It aims to describe all elements
within the railway domain.

\begin{figure}
\centering
 \includegraphics[width=\linewidth]{images/new_uml.png}
%% \includegraphics[scale=0.27]{images/new_uml.png}
 \caption{Part of \Bjoerner's DSL captured in a UML Diagram.}
\label{fig:bjoerners_dsl_pic}
\end{figure}

The UML diagram in Figure~\ref{fig:bjoerners_dsl_pic} captures
\Bjoerner's railway DSL~\cite{bjorner2003} that we will use as a
running example. It illustrates many of the features of class
diagrams:
\begin{itemize}
\item Classes, represented by a box, e.g.\ \emph{Net}, \emph{Unit},
  \emph{Station} etc. These represent concepts from the railway domain.
\item Properties, listed inside a class, e.g.\ \emph{id\,:\,UID} in the
  class \emph{Net} expresses that all \emph{Net}s have an identifier of
  type \emph{UID}.
\item Generalisations, represented by an unfilled arrow head, e.g.\
  \emph{Point} and \emph{Linear} are generalisations of \emph{Unit}.
\item Associations, represented by a line connecting two classes,
  e.g. the \emph{has} link between \emph{Unit} and \emph{Connector}.
  These can have direction, and also multiplicities associated with
  them. The multiplicities on the \emph{has} association between
  \emph{Unit} and \emph{Connector} can be read as: ``One \emph{Unit}
  has two or more connectors''.
\item Compositions, represented by a filled diamond, e.g. the
  \emph{hasLine} composition for \emph{Net} and \emph{Line}, tell us that
  one class ``is made up of'' another class. In a similar fashion to
  associations, compositions can also have multiplicities.
\item{Operations} are also represented inside a class, e.g.\ the
  \emph{isOpen} operation of type \emph{Boolean} inside the \emph{Route}
  class.
\end{itemize}

As UML class diagrams only capture static system aspects, we make the
realistic assumption that the class diagram is accompanied with some
narrative. Obviously, such a narrative can give explanations on the
static structure of the class diagram. Its main purpose, however, is
to describe the dynamic system aspects. For many domains such a
narrative is present in the form of standard literature. An example of
this is Kerr's narrative for the railway domain~\cite{kerr01}. Kerr
describes, for instance, how a signal changes using the following
narrative:
%
%\begin{quote}
  \it{''a repeater signal shows yellow if its parent signal is
    showing red''} \cite{kerr01}.
%\end{quote}
%
Further examples of narrative are \bf{N1} and \bf{N2} in Section
\ref{ssec:industry_rw}.

UML class diagrams and narratives can be linked via the stereotype
\bf{dynamic}. This annotation indicates that the labelled class
diagram element element is related to the dynamic nature of the
system. The manner in which change happens is described in the narrative. In
Figure \ref{fig:bjoerners_dsl_pic}, e.g., the relation \it{stateAt} is
marked to be of dynamic nature, i.e., to change over time. Section
\ref{sec:railway} provides the narrative how this state change
happens.

%%% Local Variables: 
%%% mode: latex
%%% TeX-master: "paper"
%%% End: 
