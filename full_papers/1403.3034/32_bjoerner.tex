\subsection{Background: \Bjoerner's DSL Adapted for ETCS}

The process of identifying, classifying and precisely defining the
elements of a domain has been coined as ``Domain Engineering'' by
\DB~\cite{bjorner2009}.  \Bjoerner gives such a classification, i.e.,
DSL, for the railway domain using a narrative~\cite{bjorner2003} which
we introduce as a running example.

Figure~\ref{fig:bjoerners_dsl_pic} shows part of the UML class diagram
for \Bjoerner's DSL. A railway is a ``Net'', built from ``Station(s)''
that are connected via ``Line(s)''.  A station can have a complex
structure, including ``Tracks'', ``Switch Points'' (also called
points) and ``Linear Units''. ``Tracks'' and ``Lines'' can only
contain ``Linear Units''. All ``Unit(s)'' are attached together via
``Connector(s)''.  Along with defining these concepts, \Bjoerner
stipulates various well-formedness conditions on such a model, for
example, ``No two distinct lines and/or stations share units.'' or
``Every line of a net is connected to exactly two, distinct
stations''~\cite{bjorner2003}.

\Bjoerner's approach contains the necessary terms to describe the
track plan in Figure~\ref{fig:trackplan}. The whole track plan forms a
\emph{station}. This \emph{station} contains all elements such as
\emph{switch points} $P1$ and $P2$ along with \emph{linear units}
$LA1, PLAT1, PLAT2$, and $LA2$, and \emph{connectors} $c1,c2,\dots,c8$
for connecting these units. From this point on, we only consider the
elements of \Bjoerner's DSL that we require for modelling track plans
we are interested in. Therefore, some of the conditions stipulated by
\Bjoerner do not apply. For example, the track plan given in
Figure~\ref{fig:trackplan} is open ended. Hence, axioms regarding
closed networks, such as the condition ``all nets must contain two
stations'' (leaving no open lines), do not apply to our models.

\Bjoerner's DSL gains dynamics by attaching a state to each
unit~\cite{bjorner2003}. Each unit can be in one of several
\emph{states} at a given time. A state is represented using a set of
paths, where a path is a pair of distinct connectors $(c,c')$. A path
expresses that a train is allowed to move along a given unit from
connector $c$ to connector $c'$. To combine a unit and a path across
it, \Bjoerner introduces \emph{unit path pairs} by forming pairs from
units and paths across them. These paths allow one to describe which
direction along a unit a train is allowed to travel. Trains are not an
explicit part of \Bjoerner's DSL. Instead, \Bjoerner describes the
concept of a ``Route'', which is a dynamic ``window'' around a
train. Concretely, routes are lists of connected units and paths
through them. For example, the route from $X$ to $PLAT1$ of
Figure~\ref{fig:trackplan} would be captured as the list
$[(LA1,(c1,c2)) , (P1,(c2,c3)), (PLAT1,(c3,c4))].$ \Bjoerner
stipulates that a route can be dynamically changed over time using a
movement function that, for a given time, gives the set of assigned
routes. This movement function extends or shrinks a route by adding or
removing units at one or both of its ends. Train movements are
modelled using this function.

Finally, \Bjoerner has formalised his narrative in (the algebraic part
of) RSL~\cite{bjorner2003}. Here we adopt the \CASL specification
language rather than RSL. Our formalisation follows in great part
\Bjoerner's modelling, however, we utilise the \CASL features of
predicates, subsorting, and structuring in order to obtain a more
readable specification text. Our choice of \CASL is also due to the
greater level of proof support that is available for \CASL in the form
of the \Hets environment~\cite{hets07}, and the institutional base for
\Hets which allows us to describe how to implement a comorphism
from UML class diagrams to \CASL in
Section~\ref{sec:comorph}. Overall, the CASL models we present capture
the narrative in a similar manner to the RSL models presented by
\Bjoerner in~\cite{bjorner2003}.

%% \subsubsection{Stereotypes for Dynamics}
%% As discuss above, the railway domain has a variety of concepts that
%% can change over time. It also has a variety of concepts that are
%% static for all time. For example, consider the track plan in
%% Figure~\ref{fig:trackplan} and the UML class diagram in
%% Figure~\ref{fig:bjoerners_dsl_pic}. Here, the \textit{has}
%% relation between Connectors and Linear units clearly remains the same
%% over the time we are interested in the system~\footnote{That is, we do
%%   not consider, for example, modelling construction changes to the
%%   railway.}. However, if we consider, for example the \textit{StateAt}
%% relation between Units and UnitStates, it is clear that the state of a
%% point may change over time. To allow UML Class Diagrams to capture
%% this dual nature of systems, we have introduced a so-called UML
%% \emph{stereotype}. In Figure~\ref{fig:bjoerners_dsl_pic} this
%% appears for the association \textit{stateAt} and the two operations
%% \textit{ isClosed\-At} and \textit{isOpen}. All other elements are
%% considered to be \textbf{rigid}. These domain-specific stereotypes are
%% intended to make clear which parts of an object structure complying to
%% the class diagram can change over time, i.e., are \textbf{dynamic}
%% like \textit{stateAt}, and which parts have to be kept fixed over
%% time, i.e., are \textbf{rigid}, e.g., the objects of \textit{Net},
%% \textit{Station}, \textit{Line}, etc. More details on the use of these
%% stereotypes can be found in ~\cite{james13}.




%%% Local Variables: 
%%% mode: latex
%%% TeX-master: "paper"
%%% End: 
