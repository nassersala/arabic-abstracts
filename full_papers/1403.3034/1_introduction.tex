Formal methods in software engineering have existed at least as long
as the term ``software engineering'' itself, which was coined at the
NATO Science Conference, Garmisch, 1968. In many engineering-based
application areas, such as in the railway domain, formal verification
processes have reached an impressive level of maturity as demonstrated
by various industrial case studies, e.g.\ see~\cite{boul00, winter02,
  winter03, peleska04, james10a}. Authors like Barnes also demonstrate
that formal methods can be cost effective~\cite{barnes11}.  Even
though these studies successfully illustrate the use of formal methods
from an academic perspective, adoption of formal methods within
industry is still limited~\cite{bowen06}.
%
From the industrial perspective, issues include
%
\begin{description}
\item[Faithful modelling] Do the proposed mathematical models
  faithfully represent the systems of concern? Modelling approaches
  offered from computer science are often in a form that is acceptable
  to computer scientists, but not to the engineer working within the
  domain. How can an engineer working within the domain come up with
  new models?

%% These presentations thus lead to doubts in the approach by
%%   the engineers and a low level of confidence towards the capabilities
%%   of the approach to correctly capture the systems being modelled.

\item[Scalability] Does the proposed technology scale up to
  industrially sized systems in a manner that is uniformly
  applicable? Often, formal methods have been applied in a pilot to
  specific systems, but require individual, hand-crafted adaptation
  and optimisation for each new system under consideration.

\item[Accessibility] Are the methods accessible to practitioners in
  the domain of interest or is it just the developers of the approach
  who can apply them? Handling of tools for verification procedures is
  often aimed towards a computer science audience specialised in
  verification, however they are usually not manageable by engineers
  outside the field of formal methods.
\end{description}
%
This paper presents a new methodology addressing these three issues. The
underlying theme is that\begin{quote}
  \centering ``Domain Specific Languages (DSLs) can aid with modelling,
  verification and encapsulation of formal methods tools within a
  given domain''.
\end{quote}

We first present our methodology in general terms and then demonstrate
it on the concrete example of verifying scheme plans from the railway
domain. Our methodology is centered around the algebraic specification
language \CASL \cite{mosses04a}. \CASL provides us with a sound
semantic foundation. Furthermore, \CASL offers mature tool support for
verification. Our methodology takes as a starting point industrial
documents describing a DSL, see for instance the Invensys Rail Data
Model~\cite{DataModel}. We then stepwise develop a modelling and
automated verification process. Thanks to elements such as graphical
tooling, the process as a whole is accessible to practitioners in the
domain, verification is scalable, and models are guaranteed to be
faithful.

The paper is organised as follows: First, we discuss our methodology
in detail. Then, we introduce railway signalling as the domain in
which we demonstrate our methodology and present the DSL which will
serve as running example throughout the paper.  The next sections
apply the methodology step by step: in Section \ref{sec:formalization}
we give a method for \emph{formalising} a DSL within \CASL (Step
M1); Section \ref{sec:verification} demonstrates how to \emph{exploit
  implicit domain knowledge} for the purpose of verification (Step
M2); Section \ref{sec:tooling} presents techniques to
\emph{encapsulate formal methods} within a tooling framework (Step M3)
-- this includes the presentation of competitive verification results
for railway scheme plan verification. Finally, we place our work in
context by discussing related work.

While in the context of this paper we deal with an ``academic'' DSL,
it is worth noting that we have successfully, i.e., with the same
positive results, applied our methodology to the DSL~\cite{DataModel}
of our industrial partner -- see \cite{james14}. The chosen academic
language is of smaller extent, however, it ``covers'' all challenging
elements.

As the paper covers such diverse topics as railways, modelling in
\CASL, verification, and tooling, we present their respective
background distributed over the paper. We use the labels
``background'' and ``contribution'' to signpost the status of each
subsection. Our paper is based upon the PhD thesis of the first
author~\cite{james14} and earlier results on the topic. Complete
specifications, further details and further examples for the work we
present in this paper can be found in the thesis~\cite{james14}. In 2011, we
published a first version of our methodology~\cite{james11a}. In 2012,
we gave a first report upon the exploitation of domain knowledge for
verification~\cite{james12}. In 2013, we discussed in detail how to
formalise DSLs within \CASL~\cite{james13}. However, this paper
comprises the first complete presentation of the whole methodology.


%% -- MR about 2 pages --

%% we concentrate on the methodology, which we present in its entirety
%% here for the first time. technical details support our constructions
%% can be found in papers ...

%% the completely new and therefore in full detail presented part is
%% the exploitation for the domain knowledge for verification.


%% Further development of the methodology published in ATE'11.
%% Integrates work published in WADT'12, HVC'12, and NFM'13. 
%% Based upon PhD thesis Swansea 2013, to appear.

%% example based -- inspired by  Bjoerners DSL for the Railway Domain
%% however, also successfully tried, e.g., on Invensys Data model

%%% Local Variables: 
%%% mode: latex
%%% TeX-master: "paper"
%%% End: 
