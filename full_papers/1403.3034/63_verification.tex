\subsection{Contribution: Verification Results}
\label{ssec:verification}
Each of the track plans in Figure~\ref{fig:plans} (TP-A to TP-D) have
been modelled using our DSL. 


\begin{figure}[h]
    \centering

 
    \subfigure[A pass through station (TP-A).]
    {       
\begin{tikzpicture}[scale=0.55,transform shape]
\node at (-0.5,0) {X};
\node at (13.5,0) {Y};
\node at (3.5,-1) {SX};
\node at (9.5,-1) {SY};
\draw [<->] (0,0.7) to (1,0.7);
\draw [<->] (13,0.7) to (12,0.7);
\RWConnector{a}{(0,0)}
\RWConnector{a1}{(1,0)}
\RWLinearUnitAbove{a}{a1}{la1}
\RWPointUpsideDown{p1}{p1cl}{p1cn}{p1cr}{(3,0)}
\RWLabelLinearUnitAbove{p1cn}{p1cl}{P1}
\RWLinearUnitAbove{a1}{p1cl}{la2}
\RWConnector{a3}{(5,0)}
\RWLinearUnitAbove{p1cn}{a3}{la3}
\RWConnector{a4}{(6,0)}
\RWLinearUnitAbove{a3}{a4}{la4}
\RWConnector{a5}{(7,0)}
\RWLinearUnitAbove{a4}{a5}{Plat}
\RWConnector{a6}{(8,0)}
\RWLinearUnitAbove{a5}{a6}{la5}
\RWPointReverseUpsideDown{p2}{p2cl}{p2cn}{p2cr}{(10,0)}
\RWLabelLinearUnitAbove{p2cn}{p2cl}{P2}
\RWLinearUnitAbove{a6}{p2cn}{la6}
\RWConnector{a7}{(12,0)}
\RWLinearUnitAbove{p2cl}{a7}{la7}
\RWConnector{a8}{(13,0)}
\RWLinearUnitAbove{a7}{a8}{la8}
\RWConnector{b1}{(5,-1)}
\RWLinearUnitBelow{p1cr}{b1}{lb1}
\RWConnector{b2}{(6,-1)}
\RWLinearUnitBelow{b1}{b2}{lb2}
\RWConnector{b3}{(7,-1)}
\RWLinearUnitBelow{b2}{b3}{lb3}
\RWConnector{b4}{(8,-1)}
\RWLinearUnitBelow{b3}{b4}{lb4}
\RWLinearUnitBelow{b4}{p2cr}{lb5}
\end{tikzpicture}
        \label{fig:first_sub}
    }
    \subfigure[A double junction (TP-B).]
    {
\begin{tikzpicture}[scale=0.55,transform shape]
\node at (-0.5,0) {A};
\node at (-0.5,-2) {B};
\node at (11.5,0) {X};
\node at (11.5,-2) {Y};
\draw [->] (0,0.7) to (1,0.7);
\draw [->] (11,-2.7) to (10,-2.7);
\draw [->] (1,-2.7) to (0,-2.7);
\draw [->] (10,0.7) to (11,0.7);
\RWConnector{a}{(0,0)}
\RWConnector{a1}{(1,0)}
\RWLinearUnitAbove{a}{a1}{la1}
\RWConnector{a2}{(2,0)}
\RWLinearUnitAbove{a1}{a2}{la2}
\RWPoint{p1}{p1cl}{p1cn}{p1cr}{(4,0)}
\RWLabelLinearUnitBelow{p1cn}{p1cl}{P1}
\RWLinearUnitAbove{a2}{p1cl}{la3}
\draw [->] ($(p1) + (0.1,0.6) $) to ($(p1cr) + (-0.3,0.2)$);
\RWPointReverseUpsideDown{p2}{p2cl}{p2cn}{p2cr}{(6,0)}
\RWLabelLinearUnitAbove{p2cn}{p2cl}{P2}
\RWPoint{p3}{p3cl}{p3cn}{p3cr}{(8,0)}
\RWLabelLinearUnitBelow{p3cn}{p3cl}{P3}
\draw [<-] ($(p3) + (0.1,0.6) $) to ($(p3cr) + (-0.3,0.2)$);
\RWConnector{a3}{(10,0)}
\RWLinearUnitAbove{p3cn}{a3}{la4}
\RWConnector{x}{(11,0)}
\RWLinearUnitAbove{a3}{x}{la5}
\RWConnector{b}{(0,-2)}
\RWConnector{b1}{(1,-2)}
\RWLinearUnitBelow{b}{b1}{lb1}
\RWConnector{b2}{(2,-2)}
\RWLinearUnitBelow{b1}{b2}{lb2}
\RWPoint{p4}{p4cl}{p4cn}{p4cr}{(4,-2)}
\RWLabelLinearUnitBelow{p4cn}{p4cl}{P4}
\RWLinearUnitBelow{b2}{p4cl}{lb3}
\RWConnector{b3}{(7,-2)}
\RWLinearUnitBelow{p4cn}{b3}{lb4}
\RWConnector{b4}{(9,-2)}
\RWLinearUnitBelow{b3}{b4}{lb5}
\RWConnector{b5}{(10,-2)}
\RWLinearUnitBelow{b4}{b5}{lb6}
\RWConnector{y}{(11,-2)}
\RWLinearUnitBelow{b5}{y}{lb7}
\end{tikzpicture}     
        \label{fig:second_sub}
    }
    \\
    \subfigure[A terminal station (TP-C).]
    {
\begin{tikzpicture}[scale=0.55,transform shape]
\node at (-0.5,0) {A};
\node at (-0.5,-1) {B};
\node at (-0.5,-3) {C};
\node at (-0.5,-4) {D};
\node at (14.5,-1) {X};
\node at (14.5,-3) {Y};
\draw [->] (14,-0.3) to (13,-0.3);
\draw [->] (13,-3.7) to (14,-3.7);
\RWConnector{a}{(0,0)}
\RWConnector{a1}{(1,0)}
\RWLinearUnitAbove{a}{a1}{PlatA}
\RWConnector{a2}{(2,0)}
\RWLinearUnitAbove{a1}{a2}{la1}
\RWPointReverse{p1}{p1cl}{p1cn}{p1cr}{(4,-1)}
\RWLabelLinearUnitBelow{p1cn}{p1cl}{P1}
\RWLinearUnitAbove{a2}{p1cr}{la2}
\RWLongPointReverseUpsideDown{p2}{p2cn}{p2cl}{p2cr}{(8,-1)}
\RWLabelLinearUnitAbove{p2cn}{p1cl}{P2}
\RWLongPointUpsideDown{p3}{p3cl}{p3cn}{p3cr}{(10,-1)}
\RWLabelLinearUnitAbove{p3cn}{p3cl}{P3}
\RWConnector{a3}{(14,-1)}
\RWLinearUnitAbove{p3cn}{a3}{la3}
\RWConnector{b}{(0,-1)}
\RWConnector{b1}{(1,-1)}
\RWLinearUnitBelow{b}{b1}{PlatB}
\RWConnector{b2}{(2,-1)}
\RWLinearUnitBelow{b1}{b2}{lb1}
\RWLinearUnitBelow{b2}{p1cn}{lb2}
\RWConnector{c}{(0,-3)}
\RWConnector{c1}{(1,-3)}
\RWLinearUnitAbove{c}{c1}{PlatC}
\RWConnector{c2}{(2,-3)}
\RWLinearUnitAbove{c1}{c2}{lc1}
\RWPointReverseUpsideDown{p4}{p4cl}{p4cn}{p4cr}{(4,-3)}
\RWLabelLinearUnitAbove{p4cn}{p4cl}{P4}
\RWLongPoint{p5}{p5cl}{p5cn}{p5cr}{(6,-3)}
\RWLabelLinearUnitBelow{p5cn}{p5cl}{P5}
\RWLinearUnitAbove{c2}{p4cn}{lc2}
\RWLongPointReverse{p6}{p6cl}{p6cn}{p6cr}{(12,-3)}
\RWLabelLinearUnitBelow{p6cn}{p6cl}{P6}
\RWConnector{y}{(14,-3)}
\RWLinearUnitBelow{p6cl}{y}{lc3}
\RWConnector{d}{(0,-4)}
\RWConnector{d1}{(1,-4)}
\RWLinearUnitBelow{d}{d1}{PlatD}
\RWConnector{d2}{(2,-4)}
\RWLinearUnitBelow{d1}{d2}{ld1}
\RWLinearUnitBelow{d2}{p4cr}{ld2}
\end{tikzpicture}
        \label{fig:third_sub}
    }
  \subfigure[A modified track plan from a London Underground
  station (TP-D).]
    {
\begin{tikzpicture}[scale=0.55,transform shape]
\node at (-0.5,0) {A};
\node at (-0.5,-2) {B};
\node at (20.5,-0) {X};
\node at (20.5,-2) {Y};
\draw [->] (0,0.7) to (1,0.7);
\draw [->] (1,-2.7) to (0,-2.7);
\RWConnector{a}{(0,0)}
\RWConnector{a1}{(1,0)}
\RWLinearUnitAbove{a}{a1}{la1}
\RWConnector{a2}{(2,0)}
\RWLinearUnitAbove{a1}{a2}{la2}
\RWConnector{a3}{(3,0)}
\RWLinearUnitAbove{a2}{a3}{la3}
\RWPointUpsideDown{p1}{p1cl}{p1cn}{p1cr}{(5,0)}
\RWLabelLinearUnitAbove{p1cl}{p1cn}{P1}
\RWLinearUnitAbove{a3}{p1cl}{la4}
\RWConnector{a4}{(8,0)}
\RWLinearUnitAbove{p1cn}{a4}{la5}
\RWConnector{a5}{(9,0)}
\RWLinearUnitAbove{a4}{a5}{la6}
\RWConnector{a6}{(10,0)}
\RWLinearUnitAbove{a5}{a6}{la7}
\RWConnector{a7}{(11,0)}
\RWLinearUnitAbove{a6}{a7}{la8}
\RWConnector{a8}{(12,0)}
\RWLinearUnitAbove{a7}{a8}{la9}
\RWPointReverseUpsideDown{p2}{p2cl}{p2cn}{p2cr}{(15,0)}
\RWLabelLinearUnitAbove{p2cl}{p2cn}{P2}
\RWLinearUnitAbove{a8}{p2cn}{la10}
\RWConnector{a9}{(17,0)}
\RWLinearUnitAbove{p2cl}{a9}{la11}
\RWConnector{a10}{(18,0)}
\RWLinearUnitAbove{a9}{a10}{la12}
\RWConnector{a11}{(19,0)}
\RWLinearUnitAbove{a10}{a11}{PlatA}
\RWConnector{a12}{(20,0)}
\RWLinearUnitAbove{a11}{a12}{la13}
\RWConnector{b}{(0,-2)}
\RWConnector{b1}{(1,-2)}
\RWLinearUnitBelow{b}{b1}{lb1}
\RWConnector{b2}{(2,-2)}
\RWLinearUnitBelow{b1}{b2}{lb2}
\RWConnector{b3}{(3,-2)}
\RWLinearUnitBelow{b2}{b3}{lb3}
\RWConnector{b4}{(4,-2)}
\RWLinearUnitBelow{b3}{b4}{lb4}
\RWPointReverse{p3}{p3cl}{p3cn}{p3cr}{(7,-2)}
\RWLabelLinearUnitBelow{p3cl}{p3cn}{P3}
\RWLinearUnitBelow{b4}{p3cn}{lb5}
\RWConnector{b5}{(9,-2)}
\RWLinearUnitBelow{p3cl}{b5}{lb6}
\RWConnector{b6}{(10,-2)}
\RWLinearUnitBelow{b5}{b6}{lb7}
\RWConnector{b7}{(11,-2)}
\RWLinearUnitBelow{b6}{b7}{lb8}
\RWPoint{p4}{p4cl}{p4cn}{p4cr}{(13,-2)}
\RWLabelLinearUnitBelow{p4cl}{p4cn}{P4}
\RWLinearUnitBelow{b7}{p4cl}{lb9}
\RWConnector{b8}{(16,-2)}
\RWLinearUnitBelow{p4cn}{b8}{lb10}
\RWConnector{b9}{(17,-2)}
\RWLinearUnitBelow{b8}{b9}{lb11}
\RWConnector{b10}{(18,-2)}
\RWLinearUnitBelow{b9}{b10}{lb12}
\RWConnector{b11}{(19,-2)}
\RWLinearUnitBelow{b10}{b11}{PlatB}
\RWConnector{b12}{(20,-2)}
\RWLinearUnitBelow{b11}{b12}{lb13}
\end{tikzpicture}
}
    \caption{Verified track plans.}
    \label{fig:plans}
\end{figure}

We have then included a then \THEN {\small{}\KW{\%}\KW{implies}} block
for the proof of safety. This block is also generated automatically by
the OnTrack tool, as all information required for this block is
available from the graphical model. The block is structured in two
parts, the first contains lemmas to be proven, then the second our
overall proof goal. The aim of the lemmas from the first block is to
encode a case distinction used in the proof of safety. That is, they
help the automated prover in proving our final goal. For verification
using our DSL, these lemmas are in the form of our final proof goal
instantiated for each route of the scheme plan under
consideration. That is, for a route R, we have:
\begin{small}
\begin{hetcasl}
\> \Ax{\forall} \Id{t} \Ax{:} \Id{Time}; \Id{rg} \Ax{:} \IdApplLabel{\Id{Region}}{Region}; \Id{ma} \Ax{:} \Id{MA} \\
\>\> \Ax{\bullet} \=\IdApplLabel{\Id{assigned}}{assigned}(\=\Id{ma}, \Id{t}) \Ax{\wedge} \=\Id{rg} \IdApplLabel{\Id{eps}}{::eps::} \Id{ma} \Ax{\wedge} \=\Id{rg} \IdApplLabel{\Id{eps}}{::eps::} \IdApplLabel{\Id{regions}}{regions}(\Id{R}) 
 \Ax{\Rightarrow} \Ax{\neg} \=\Id{R} \IdApplLabel{\Id{isOpenAt}}{::isOpenAt::} \Id{t} 
\end{hetcasl}
\end{small}



\noindent Once the above implied axioms have been proven, they can be used to
help deduce our overall proof goal:
\begin{small}
\begin{hetcasl}
\THEN \={\small{}\KW{\%}\KW{implies}}\\
\> \Ax{\forall} \Id{t} \Ax{:} \Id{Time}; \Id{r} \Ax{:} \IdApplLabel{\Id{Route}}{Route}; \Id{rg} \Ax{:} \IdApplLabel{\Id{Region}}{Region}; \Id{ma} \Ax{:} \Id{MA} \\
\>\> \Ax{\bullet} \=\IdApplLabel{\Id{assigned}}{assigned}(\=\Id{ma}, \Id{t}) \Ax{\wedge} \=\Id{rg} \IdApplLabel{\Id{eps}}{::eps::} \Id{ma} \Ax{\wedge} \=\Id{rg} \IdApplLabel{\Id{eps}}{::eps::} \IdApplLabel{\Id{regions}}{regions}(\Id{r}) 
\Ax{\Rightarrow} \Ax{\neg} \=\Id{r} \IdApplLabel{\Id{isOpenAt}}{::isOpenAt::} \Id{t} 
\end{hetcasl}
\end{small}

The verification times presented in Figure~\ref{fig:times} show that
verification is possible over our enriched DSL. The average memory
column shows the average memory used across all route lemma proofs and
the safety proof. The proofs have been performed on a quad core $3$GHz
machine with $8$GB of RAM running Ubuntu 12.04.

\begin{figure}[h]
\begin{small}
\centering
\begin{tabular}{|l|r|r|r|}
\hline
Track Plan & Routes Lemma Proofs (s) & Safety Proof (s) & Avg. Memory (MB)\\
\hline  \hline

TP-A &  54.54  & 10.04 & 176.88 \\
TP-B &  23.88  & 6.57 & 90.36 \\
TP-C &  194.41  & 22.34 & 188.07 \\
TP-D &  236.10  & 20.46 & 489.64 \\
\hline  
\end{tabular}
\end{small}
\caption{Verification times for the given track plans.}
\label{fig:times}
\end{figure}

All proofs are completed relatively quickly, with the longest proof
time being for track plan TP-D. Interestingly, this is due to the
number of units contained within this track plan. Also, it is
interesting to note that the track plans that look more complicated
and contain more possible routes, i.e.\ TP-B and TP-C are relatively
quick to verify. This is because these track plans get split into many
smaller regions compared with the fewer larger regions of track plan
TP-D. Each of these small regions is represented by a list containing
fewer elements within our modelling. Hence, the automated theorem
prover does not need to search to such a depth to find a proof. This
point is also illustrated by the increased average memory usage for
TP-D. This shows that the DSL lemmas we have introduced give a
measurable effect for verification. We note that these times
outperform the typical times for model checking on such stations as,
e.g., as presented in \cite{MNRST12HVC}.

%%% Local Variables: 
%%% mode: latex
%%% TeX-master: "paper"
%%% End: 
