\section{Specification Structure}
\label{app:struture}
Throughout the paper, we have introduced various levels of
specification that are structured as follows:


\begin{small}
\begin{hetcasl}
\>\SPEC \SId{DSLForVerification} \Ax{=} \\
\>\> \SId{Datatypes} \\
\>\THEN \\
\>\> \SId{DSL}\\
\> \THEN\\
\>\> \SId{DSLExtension}\\
\>\THEN \={\small{}\KW{\%}\KW{implies}}\\
\>\>  \SId{DSLemmas} \\
\>\THEN \\
\>\> \SId{ConcreteSchemePlan}\\
\>\THEN \={\small{}\KW{\%}\KW{implies}}\\
\>\> \SId{Safety}\\
\>\KW{end}
\end{hetcasl}
\end{small}

\noindent Here we can see, that our specifications begin with the Datatypes
(\SId{Datatypes}) and DSL (\SId{DSL}) gained from our translation of
the UML class diagram. We then extend these specifications with
\SId{DSLExtension} for modelling the narrative aspects of the original
informal DSL. Concretely this includes our modelling of movement
authorities. Next, we can see that the property supporting lemmas
(\SId{DSLemmas}) are added to aid with verification. These are added
as implied axioms over the extended DSL. This illustrates that these
lemmas are independent of any scheme plan formulated in the DSL. Next,
we see the specification \SId{ConcreteSchemePlan} that encodes a
particular track plan and its associated movement authorities and
control tables. Finally, we can see the proof goals for proving safety
are added (\SId{Safety}). These are again added as implied axioms, and
are then proven relative to the given concrete scheme plan.

%%% Local Variables: 
%%% mode: latex
%%% TeX-master: "paper"
%%% End: 
