\ifarxiv
    \newcommand{\reftabFID}{Table~\ref{tab:FID}}
    \newcommand{\shortreftabFID}{Table~\ref{tab:FID}}
    \newcommand{\refsecseparability}{Section~\ref{sec:separability}}
    \newcommand{\refsectruncation}{Appendix~\ref{sec:truncation}}
\else
    \newcommand{\reftabFID}{Table~1 of the paper}
    \newcommand{\shortreftabFID}{Table~1}
    \newcommand{\refsecseparability}{Section~4.2 of the paper}
    \newcommand{\refsectruncation}{Appendix~B}
\fi

\section{Hyperparameters and training details}
\label{sec:hyperparams}

We build upon the official TensorFlow~\cite{Tensorflow} implementation of Progressive GANs by Karras~et~al.~\cite{Karras2017}, from which we inherit most of the training details.\footnote{\texttt{\scriptsize https://github.com/tkarras/progressive\_growing\_of\_gans}} This original setup corresponds to configuration \arch{a} in \reftabFID.
In particular, we use the same \FINAL{discriminator architecture}, resolution-dependent minibatch sizes, Adam~\cite{Adam} hyperparameters, \FINAL{and exponential moving average of the generator}. We enable mirror augmentation for CelebA-HQ and FFHQ, \FINAL{but disable it for LSUN}.
Our training time is approximately one week on an NVIDIA DGX-1 with 8 Tesla V100 GPUs.

For our improved baseline (\arch{b} in \shortreftabFID), we make several modifications to improve the overall result quality.
We replace the nearest-neighbor up/downsampling in both networks with bilinear sampling, which we implement by low-pass filtering the activations with a separable 2\textsuperscript{nd} order binomial filter after each upsampling layer and before each downsampling \FINAL{layer \cite{zhang2019}}.
We implement progressive growing the same way as Karras~et~al.~\cite{Karras2017}, but we start from $8^2$ images instead of $4^2$.
For the FFHQ dataset, we switch from WGAN-GP to the non-saturating loss \cite{Goodfellow2014} with $R_1$ regularization \cite{Mescheder2018} using $\gamma=10$.
With $R_1$ we found that the FID scores keep decreasing for considerably longer than with WGAN-GP, and we thus increase the training time from 12M to 25M images.
We use the same learning rates as Karras~et~al.~\cite{Karras2017} for FFHQ, but we found that setting the learning rate to 0.002 instead of 0.003 for $512^2$ and $1024^2$ leads to better stability with CelebA-HQ.

For our style-based generator (\arch{f} in \shortreftabFID), we use leaky ReLU \cite{Maas2013} with $\alpha=0.2$ and equalized learning rate~\cite{Karras2017} for all layers.
We use the same feature map counts in our convolution layers as Karras~et~al.~\cite{Karras2017}.
Our mapping network consists of 8 fully-connected layers, and the dimensionality of all input and output activations\,---\,including $\zz$ and $\ww$\,---\,is 512.
We found that increasing the depth of the mapping network tends to make the training unstable with high learning rates.
We thus reduce the learning rate by two orders of magnitude for the mapping network, i.e., $\lambda' = 0.01\cdot\lambda$.
We initialize all weights of the convolutional, fully-connected, and affine transform layers using $\mathcal{N}(0,1)$.
The constant input in synthesis network is initialized to one.
The biases and noise scaling factors are initialized to zero, except for the biases associated with $\yy_s$ that we initialize to one.

The classifiers used by our separability metric (\refsecseparability) have the same architecture as our discriminator except that minibatch standard deviation~\cite{Karras2017} is disabled.
We use the learning rate of $10^{-3}$, minibatch size of 8, Adam optimizer, and training length of 150,000 images.
The classifiers are trained independently of generators, and the same~40 classifiers, one for each CelebA attribute, are used for measuring the separability metric for all generators.
We will release the pre-trained classifier networks so that our measurements can be reproduced.

\figpathlenbehavior

We do not use batch normalization \cite{Ioffe2015}, spectral normalization \cite{Miyato2018B}, attention mechanisms \cite{Zhang2018sagan}, dropout \cite{srivastava2014}, or pixelwise feature vector normalization \cite{Karras2017} in our networks.

\section{Training convergence}
\figqualitybedroom

Figure~\ref{fig:pathlenbehavior} shows how the FID and perceptual path length metrics evolve during the training of our configurations \arch{b} and \arch{f} with the FFHQ dataset.
With $R_1$ regularization active in both configurations, FID continues to slowly decrease as the training progresses, 
motivating our choice to increase the training time from 12M images to 25M images.
Even when the training has reached the full $1024^2$ resolution, the slowly rising path lengths indicate that the improvements in FID come at the cost of a more entangled representation.
Considering future work, it is an interesting question whether this is unavoidable, or if it were possible to encourage shorter path lengths without compromising the convergence of FID.

\section{Other datasets}
\label{sec:otherdatasets}

Figures~\ref{fig:qualitybedroom},~\ref{fig:qualitycar}, and~\ref{fig:qualitycat} show an uncurated set of results for \textsc{LSUN} \cite{LSUN} \textsc{Bedroom}, \textsc{Cars}, and \textsc{Cats}, respectively. In these images we used the truncation trick from \refsectruncation with $\psi=0.7$ for resolutions $4^2-32^2$. The accompanying video provides results for style mixing and stochastic variation tests. As can be seen therein, in case of \textsc{Bedroom} the coarse styles basically control the viewpoint of the camera, middle styles select the particular furniture, and fine styles deal with colors and smaller details of materials. In \textsc{Cars} the effects are roughly similar. Stochastic variation affects primarily the fabrics in \textsc{Bedroom}, backgrounds and headlamps in \textsc{Cars}, and fur, background, and interestingly, the positioning of paws in \textsc{Cats}. Somewhat surprisingly the wheels of a car never seem to rotate based on stochastic inputs.

These datasets were trained using the same setup as FFHQ for the duration of 70M images for \textsc{Bedroom} and \textsc{Cats}, and 46M for \textsc{Cars}. We suspect that the results for \textsc{Bedroom} are starting to  approach the limits of the training data, as in many images the most objectionable issues are the severe compression artifacts that have been inherited from the low-quality training data. \textsc{Cars} has much higher quality training data that also allows higher spatial resolution ($512\times384$ instead of $256^2$), and \textsc{Cats} continues to be a difficult dataset due to the high intrinsic variation in poses, zoom levels, and backgrounds.

\figqualitycar
\figqualitycat
